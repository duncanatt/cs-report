%!TEX root=main.tex
%!TEX spellcheck=en_GB
%!TEX program=pdflatex

% Include preamble.
%!TEX root=main.tex
\documentclass[11pt,a4paper,draft,abref]{csreport}

%:::::::::::::::::::::::::::::::::::::::::::::::::::::::::::::::::::::::::::::::
% Package imports with basic inlined configuration.
%:::::::::::::::::::::::::::::::::::::::::::::::::::::::::::::::::::::::::::::::

% Provides automated and sensible handling of trailing spaces which eats or 
% inserts spaces depending on the command context stream that follows.
\usepackage{xspace}

% Provides extended support for the array and tabular environments for optional 
% column formats.
\usepackage{array}

% Provides support for multirow tables.
\usepackage{multirow}

% Provides miscellaneous mathematical symbols (such as denotation brackets).
\usepackage{stmaryrd}

% Provides support for modifying standard enumerations (such as inline lists,
% roman numerals in lists, etc.).
\usepackage[inline]{enumitem}

% Provides miscellaneous dingbats (such as check marks, ticks, etc.).
\usepackage{pifont}

% Provides support for highlighting text.
\usepackage{soul}

% Provides support for low-level string processing (such as string chopping).
\usepackage{xstring}

% Provides support for including and displaying code listings for various 
% languages.
\usepackage{listings}


%:::::::::::::::::::::::::::::::::::::::::::::::::::::::::::::::::::::::::::::::
% Additional package-specific configuration.
%:::::::::::::::::::::::::::::::::::::::::::::::::::::::::::::::::::::::::::::::


% Includes.
%:::::::::::::::::::::::::::::::::::::::::::::::::::::::::::::::::::::::::::::::
% Math lettering shorthand.
%:::::::::::::::::::::::::::::::::::::::::::::::::::::::::::::::::::::::::::::::

% Shorthand for math calligraphic letters.
\newcommand{\cA}{\ensuremath{\mathcal{A}}\xspace}
\newcommand{\cB}{\ensuremath{\mathcal{B}}\xspace}
\newcommand{\cC}{\ensuremath{\mathcal{C}}\xspace}
\newcommand{\cD}{\ensuremath{\mathcal{D}}\xspace}
\newcommand{\cE}{\ensuremath{\mathcal{E}}\xspace}
\newcommand{\cF}{\ensuremath{\mathcal{F}}\xspace}
\newcommand{\cG}{\ensuremath{\mathcal{G}}\xspace}
\newcommand{\cH}{\ensuremath{\mathcal{H}}\xspace}
\newcommand{\cI}{\ensuremath{\mathcal{I}}\xspace}
\newcommand{\cJ}{\ensuremath{\mathcal{J}}\xspace}
\newcommand{\cK}{\ensuremath{\mathcal{K}}\xspace}
\newcommand{\cL}{\ensuremath{\mathcal{L}}\xspace}
\newcommand{\cM}{\ensuremath{\mathcal{M}}\xspace}
\newcommand{\cN}{\ensuremath{\mathcal{N}}\xspace}
\newcommand{\cO}{\ensuremath{\mathcal{O}}\xspace}
\newcommand{\cP}{\ensuremath{\mathcal{P}}\xspace}
\newcommand{\cQ}{\ensuremath{\mathcal{Q}}\xspace}
\newcommand{\cR}{\ensuremath{\mathcal{R}}\xspace}
\newcommand{\cS}{\ensuremath{\mathcal{S}}\xspace}
\newcommand{\cT}{\ensuremath{\mathcal{T}}\xspace}
\newcommand{\cU}{\ensuremath{\mathcal{U}}\xspace}
\newcommand{\cV}{\ensuremath{\mathcal{V}}\xspace}
\newcommand{\cW}{\ensuremath{\mathcal{W}}\xspace}
\newcommand{\cX}{\ensuremath{\mathcal{X}}\xspace}
\newcommand{\cY}{\ensuremath{\mathcal{Y}}\xspace}
\newcommand{\cZ}{\ensuremath{\mathcal{Z}}\xspace}

% Shorthand for math sans-serif letters.
\newcommand{\sA}{\ensuremath{\mathsf{A}}\xspace}
\newcommand{\sB}{\ensuremath{\mathsf{B}}\xspace}
\newcommand{\sC}{\ensuremath{\mathsf{C}}\xspace}
\newcommand{\sD}{\ensuremath{\mathsf{D}}\xspace}
\newcommand{\sE}{\ensuremath{\mathsf{E}}\xspace}
\newcommand{\sF}{\ensuremath{\mathsf{F}}\xspace}
\newcommand{\sG}{\ensuremath{\mathsf{G}}\xspace}
\newcommand{\sH}{\ensuremath{\mathsf{H}}\xspace}
\newcommand{\sI}{\ensuremath{\mathsf{I}}\xspace}
\newcommand{\sJ}{\ensuremath{\mathsf{J}}\xspace}
\newcommand{\sK}{\ensuremath{\mathsf{K}}\xspace}
\newcommand{\sL}{\ensuremath{\mathsf{L}}\xspace}
\newcommand{\sM}{\ensuremath{\mathsf{M}}\xspace}
\newcommand{\sN}{\ensuremath{\mathsf{N}}\xspace}
\newcommand{\sO}{\ensuremath{\mathsf{O}}\xspace}
\newcommand{\sP}{\ensuremath{\mathsf{P}}\xspace}
\newcommand{\sQ}{\ensuremath{\mathsf{Q}}\xspace}
\newcommand{\sR}{\ensuremath{\mathsf{R}}\xspace}
\newcommand{\sS}{\ensuremath{\mathsf{S}}\xspace}
\newcommand{\sT}{\ensuremath{\mathsf{T}}\xspace}
\newcommand{\sU}{\ensuremath{\mathsf{U}}\xspace}
\newcommand{\sV}{\ensuremath{\mathsf{V}}\xspace}
\newcommand{\sW}{\ensuremath{\mathsf{W}}\xspace}
\newcommand{\sX}{\ensuremath{\mathsf{X}}\xspace}
\newcommand{\sY}{\ensuremath{\mathsf{Y}}\xspace}
\newcommand{\sZ}{\ensuremath{\mathsf{Z}}\xspace}

% Shorthand for math blackboard bold letters (requires amsfonts package).
\newcommand{\bA}{\ensuremath{\mathbb{A}}\xspace}
\newcommand{\bB}{\ensuremath{\mathbb{B}}\xspace}
\newcommand{\bC}{\ensuremath{\mathbb{C}}\xspace}
\newcommand{\bD}{\ensuremath{\mathbb{D}}\xspace}
\newcommand{\bE}{\ensuremath{\mathbb{E}}\xspace}
\newcommand{\bF}{\ensuremath{\mathbb{F}}\xspace}
\newcommand{\bG}{\ensuremath{\mathbb{G}}\xspace}
\newcommand{\bH}{\ensuremath{\mathbb{H}}\xspace}
\newcommand{\bI}{\ensuremath{\mathbb{I}}\xspace}
\newcommand{\bJ}{\ensuremath{\mathbb{J}}\xspace}
\newcommand{\bK}{\ensuremath{\mathbb{K}}\xspace}
\newcommand{\bL}{\ensuremath{\mathbb{L}}\xspace}
\newcommand{\bM}{\ensuremath{\mathbb{M}}\xspace}
\newcommand{\bN}{\ensuremath{\mathbb{N}}\xspace}
\newcommand{\bO}{\ensuremath{\mathbb{O}}\xspace}
\newcommand{\bP}{\ensuremath{\mathbb{P}}\xspace}
\newcommand{\bQ}{\ensuremath{\mathbb{Q}}\xspace}
\newcommand{\bR}{\ensuremath{\mathbb{R}}\xspace}
\newcommand{\bS}{\ensuremath{\mathbb{S}}\xspace}
\newcommand{\bT}{\ensuremath{\mathbb{T}}\xspace}
\newcommand{\bU}{\ensuremath{\mathbb{U}}\xspace}
\newcommand{\bV}{\ensuremath{\mathbb{V}}\xspace}
\newcommand{\bW}{\ensuremath{\mathbb{W}}\xspace}
\newcommand{\bX}{\ensuremath{\mathbb{X}}\xspace}
\newcommand{\bY}{\ensuremath{\mathbb{Y}}\xspace}
\newcommand{\bZ}{\ensuremath{\mathbb{Z}}\xspace}


%:::::::::::::::::::::::::::::::::::::::::::::::::::::::::::::::::::::::::::::::
% Math object definitions.
%:::::::::::::::::::::::::::::::::::::::::::::::::::::::::::::::::::::::::::::::

% Operators.
\DeclareMathOperator{\righttriplearrows} {{\; \tikz{ \foreach \y in {0, 0.1, 0.2} { \draw [-stealth] (0, \y) -- +(0.5, 0);}} \; }}
\newcommand{\Righttriplearrows} {{\; \tikz{ \foreach \y in {0, 0.1, 0.2} { \draw [-stealth] (0, \y) -- +(0.5, 0);}} \; }}

% Actions and traces.
\makeatletter
\newcommand{\actiont}[1]{%
	\@tempswafalse
	\@for\next:=#1\do
	{\if@tempswa.\else\@tempswatrue\fi\texttt{\next}}%
}
\newcommand{\trace}[1]{\textit{\actiont{#1}}}
\makeatother

% Process constants.
\newcommand{\kP}[1][]{\if\relax\detokenize{#1}\relax\ensuremath{P}\else\ensuremath{P_{#1}}\fi\xspace}
\newcommand{\kQ}[1][]{\if\relax\detokenize{#1}\relax\ensuremath{Q}\else\ensuremath{Q_{#1}}\fi\xspace}
\newcommand*{\kSys}{\ensuremath{\textsc{Sys}}\xspace}


%:::::::::::::::::::::::::::::::::::::::::::::::::::::::::::::::::::::::::::::::
% Environments.
%:::::::::::::::::::::::::::::::::::::::::::::::::::::::::::::::::::::::::::::::

% Define the general environment 'QED' marker.
\newcommand*{\envqed}{\hfill$\blacksquare$}

% Checks if a counter exists.
\makeatletter
\newcommand*\ifcounter[1]{%
  \ifcsname c@#1\endcsname
    \expandafter\@firstoftwo
  \else
    \expandafter\@secondoftwo
  \fi
}

\ifcounter{chapter}{
  
  % Example environment.
  \newcounter{example}[chapter]
  \renewcommand{\theexample}{\thechapter.\arabic{example}}
  \newenvironment{example}[1][]{
    \refstepcounter{example}\bigskip
    \if\relax\detokenize{#1}\relax
      \noindent\textbf{Example~\theexample}.
    \else
      \noindent\textbf{Example~\theexample} (#1).
    \fi}
  {
    \envqed\bigskip
  }

  % Definition environment.
  \newcounter{definition}[chapter]
  \renewcommand{\thedefinition}{\thechapter.\arabic{definition}}
  \newenvironment{definition}[1][]{
    \refstepcounter{definition}\bigskip
    \if\relax\detokenize{#1}\relax
      \noindent\textbf{Definition~\thedefinition}.
    \else
      \noindent\textbf{Definition~\thedefinition} (#1).
    \fi}
  {
    \envqed\bigskip
  }

  % Theorem environment.
  \newcounter{theorem}[chapter]
  \renewcommand{\thetheorem}{\thechapter.\arabic{theorem}}
  \newenvironment{theorem}[1][]{
    \refstepcounter{theorem}\bigskip
    \if\relax\detokenize{#1}\relax
      \noindent\textbf{Theorem~\thetheorem}.
    \else
      \noindent\textbf{Theorem~\thetheorem} (#1).
    \fi}
  {
    \envqed\bigskip
  }
}
\makeatother
%!TEX root=../main.tex

% Abbreviations.
\newcommand*{\eg}{\textit{e.g.}\xspace}
\newcommand*{\Eg}{\textit{E.g.}\xspace}
\newcommand*{\ie}{\textit{i.e.,}\xspace}
\newcommand*{\etc}{\textit{etc.}\xspace}
\newcommand*{\corr}{\textit{corr.}\xspace}
\newcommand*{\etal}{\textit{et~al.}\xspace}
\newcommand*{\wrt}{w.r.t.\xspace}
\newcommand*{\cf}{\textit{cf.}\xspace}
\newcommand*{\nb}{\textit{n.b.}\xspace}
\newcommand*{\resp}{resp.\xspace}
\newcommand*{\vs}{\textit{vs.}\xspace}

% Quotations.
\newcommand{\quot}[1]{\textit{``#1''}}

% Acronyms.
\newcommand*{\PID}{PID\xspace}
\newcommand*{\RV}{RV\xspace}
\newcommand*{\LTL}{\textsc{LTL}\xspace}
\newcommand*{\PtDTL}{\textsc{PtDTL}\xspace}
\newcommand*{\DTL}{\textsc{DTL}\xspace}
\newcommand*{\LTLIII}{\ensuremath{\textsc{LTL}_3}\xspace}
\newcommand*{\LTLK}{\ensuremath{\textsc{LTL}_{2k+4}}\xspace}
\newcommand*{\CTL}{CTL\xspace}
\newcommand*{\muHML}{\ensuremath{\mu}HML\xspace}
\newcommand*{\mHML}{\textsc{mHML}\xspace}
\newcommand*{\sHML}{\textsc{sHML}\xspace}
\newcommand*{\DATE}{\textsc{DATE}\xspace}
\newcommand*{\MtTL}{\textsc{MtTL}\xspace}
\newcommand*{\MTL}{\textsc{MTL}\xspace}
\newcommand*{\muCalculus}{\ensuremath{\mu}-calculus\xspace}
\newcommand*{\AOP}{AOP\xspace}
\newcommand*{\Erlang}{Erlang\xspace}
\newcommand*{\BIF}{BIF\xspace}
%!TEX root=../main.tex

%:::::::::::::::::::::::::::::::::::::::::::::::::::::::::::::::::::::::::::::::
% Additional TikZ configuration.
%:::::::::::::::::::::::::::::::::::::::::::::::::::::::::::::::::::::::::::::::

% Imported TikZ libraries.
\usetikzlibrary {
  matrix,
  shapes,
  arrows,
  shadows,
  calc,
  chains,
  decorations.pathmorphing,
  decorations.text,
  arrows.meta,
  patterns,
  fit
}


%:::::::::::::::::::::::::::::::::::::::::::::::::::::::::::::::::::::::::::::::
% Graphic objects.
%:::::::::::::::::::::::::::::::::::::::::::::::::::::::::::::::::::::::::::::::

% Custom shapes.
\tikzset{
  label/.style={
    font=\scriptsize\itshape,
    inner sep=0.4em
  },
  point/.style={
    circle,
    fill=black,
    text width=0.3em,
    inner sep=0
  },
  state/.style={
    circle,
    text width=0.8em,
    inner sep=0.1em,
    text depth=0.08em,
    draw,
    font=\scriptsize
  }
}

% Set title and subtitle
\title{Ensuring Correctness in\\Distributed Systems}
\subtitle{An example report}

% Set document coordinates and date.
\author{Duncan Paul Attard}
\department{Department of Computer Science}
\faculty{Faculty of ICT}
\supervisors{Adrian Francalanza\\Luca Aceto\\Anna Ing\'olfsd\'ottir}
\date{\today}

\begin{document}

  % Make title page.
  \maketitle

  \begin{abstract}
    
    A number of software systems today are built in terms of independently executing components that typically reside on different physical locations.
    %
    While these software organisations offer a number of advantages, including the use of replication to improve robustness and quality of service, they are hard to design and implement. 
    %
    Ascertaining their correctness, therefore, becomes a chief concern.
    %
    Traditional formal verification techniques, such as testing or model checking, tend to be applied with limited success in these scenarios due to a number of reasons.
    %
    Runtime verification may be employed as a lightweight and dynamic alternative that complements the aforementioned verification approaches.
    
    \ldots
  \end{abstract}


  % Disable protrusion for TOC, list of figures etc. so that certain 
  % numbers (e.g. 7) do not spill out of the boundary of the right margin but 
  % remain on the same margin as the rest of the other numbers.
  \microtypesetup{protrusion=false} 
  
  % Switch to roman page numbering until main manuscript body.
  \pagenumbering{roman}

  % Configure and show TOC and LOF.
  \setcounter{tocdepth}{2}
  \tableofcontents
  \addtocontents{toc}{~\hfill\textbf{Page}\par}
  \listoffigures
  
  % Re-enable protrusion as this is typographically nice to have for the rest of 
  % the text in the manuscript.
  \microtypesetup{protrusion=true}
  \clearpage

  % Switch page numbering to roman and include main manuscript content.
  \pagenumbering{arabic}

  %!TEX root=main.tex
\chapter{Introduction}\label{ch:introduction}

Numerous software systems are nowadays architected in terms of \emph{asynchronous components}~\cite{Chappell2004,Josuttis2007,AghaMST97} that execute independently to one another without recourse to a global clock or shared state.
%
Instead, components interact together via well-defined interfaces and non-blocking messaging~\cite{HohpeWoolf2003} to create dynamic and loosely-coupled software organisations.
%
Such architectures facilitate incremental updates, tolerate independent component failures and permit the various units of execution to be \emph{distributed} across different locations~\cite{Garg2014,DollimoreKindbergCoulouris2005}. 
%
Despite their advantages, these systems are notoriously hard to design, and even harder to program and get right, and ensuring their correctness in terms of their expected behaviour becomes paramount.

\section{Distributed Runtime Verification}

While distributed systems inherit the characteristics inherent to asynchronous settings, their execution is further complicated due to physical constraints, such as the lack of a global clock and possibility of independent failures~\cite{Ghosh2014,DollimoreKindbergCoulouris2005}.
%
In the local asynchronous case, traditional pre-deployment verification techniques such as model checking and testing~\cite{ClarkeGrumbergPeled1999,MyersSandlerBadgett2011} often \emph{scale poorly} because the set of execution paths considered is invariably dwarfed by the vast number of possible execution paths of the system.
%
In a distributed scenario, use of these verification approaches is often problematic, if not \emph{impractical}, due to the aforementioned complications. 

\emph{Runtime Verification} (\RV)~\cite{LeuckerS09,FalconeFM12} is complementary approach that evades some of the limitations of pre-deployment techniques by deferring the analysis until runtime.
%
It employs \emph{monitors}
%
to \emph{incrementally} analyse the system's behaviour (exhibited as a sequence of \emph{trace} events) up to the current execution point, to determine whether a correctness specification under investigation is satisfied or violated.

\ldots
 
\begin{example}\label{eg:local}
  Consider the system \kSys composed of processes \kP and \kQ, given in terms of the \emph{labelled transition systems} in \cref{fig:lts}.
  %
  \kP can initially fork (denoted by action \actiont{f}) a process, after which it either sends (\actiont{s}) a message and exits (\actiont{e}), or exits (\actiont{e}) immediately; \kQ is a simpler process that performs a single receive (\actiont{r}) and exits (\actiont{e}).
  %
  A possible correctness property states that \quot{\kSys does not fork processes at startup}.

  \medskip
  When \kSys exhibits the witness trace \trace{f,r,e,e}, the monitor can detect a violation of this property.
  %
  For a different execution interleaving, \eg \trace{r,e,f,e} (where \trace{f} is not the first event), the typical \RV analysis would be unable to detect the fact that \kSys is capable of performing \actiont{f}.
  %
  Recouping this \emph{lack of precision} is possible, but this would require the specification to consider \emph{all} the possible trace event permutations that the composition of \kP and \kQ may exhibit (\cref{fig:lts_sys}). 
  %
  One easily observes that adding new components to \kSys aggravates the specification task to the point where it becomes unwieldy and error-prone.
  Reformulating the original property to consider \kP in isolation, \ie \quot{\kP does not fork processes at startup}, eliminates the need to account for the behaviour of other nonrelevant components.
\end{example}

\begin{figure}[t]
    \centering
    \subfloat[The labelled transition systems for processes \kP and \kQ\label{fig:lts_pq}]{
      \begin{tikzpicture}[>=stealth', shorten >=0.05em, shorten <=0.05em, auto, node distance=2em]
        \draw[shift={(-5.5em,0)}, color=white] (0em,0em) rectangle (16em,1em);

        % Draw process P.
        \node[point] (p0) {};
        \node[point, below=of p0] (p1) {};
        \node[point, below=of p1, xshift=2em] (p2) {};
        \node[point, below=of p1, xshift=-2em] (p3) {};
        \node[label, above=0 of p0]{\textcolor{white}{\kQ}\!\!\!\!\!\kP};

        % Draw process Q.
        \node[point, right=6em of p0] (q0) {};
        \node[point, below=of q0] (q1) {};
        \node[point, below=of q1] (q2) {};
        \node[label, above=0 of q0]{\kQ};

        % Draw transitions between P nodes.
        \path[-stealth', font=\tiny\ttfamily, left]
          (p0) edge [line width=1pt] node [xshift=0.1em] {f} (p1)
          (p1) edge node [right, xshift=0.1em, pos=0.3] {e} (p2)
          (p1) edge node [pos=0.3] {s} (p3)
          (p3) edge [out=330, in=210] node [below] {e} (p2);
        
        % Draw transitions between Q nodes.
        \path[-stealth', font=\tiny\ttfamily, left]
          (q0) edge node [xshift=0.1em] {r} (q1)
          (q1) edge node [xshift=0.1em] {e} (q2);
      \end{tikzpicture}
    }\qquad\qquad
    \subfloat[System \kSys as a composition of processes \kP and \kQ\label{fig:lts_sys}]{
      \begin{tikzpicture}[>=stealth', shorten >=0.05em, shorten <=0.05em, auto, node distance=2em]
        \draw[shift={(-9em,0)}, color=white] (0em,0em) rectangle (16em,1em);

        % Draw composed system.
        % \node[point, right=10em of q0] (s0) {};
        \node[point] (s0) {};
        \node[point, below=of s0, xshift=2em] (s1) {};
        \node[point, below=of s0, xshift=-2em] (s2) {};
        \node[point, below=of s2, xshift=2em] (s3) {};
        \node[point, below=of s2, xshift=-2em] (s4) {};
        \node[point, below=of s1, xshift=2em] (s5) {};
        \node[point, below=of s3, xshift=-2em] (s6) {};
        \node[point, below=of s5] (s7) {};
        \node[point, below=2em of s6, xshift=2em] (s8) {};
        \node[point, below=4em of s6, xshift=-2em] (s9) {};
        \node[point, below=of s7] (s10) {};
        \node[label, above=0 of s0]{$\kSys=\kP\!\mid\!\kQ$};

        % Draw transitions between system nodes.
        \path[-stealth', font=\tiny\ttfamily, left]
          (s0) edge [line width=1pt] node [right, xshift=0.1em, pos=0.3] {f} (s1)
          (s0) edge node [pos=0.3] {r} (s2)
          (s1) edge node [pos=0.3] {r} (s3)
          (s2) edge [line width=1pt] node [right, xshift=0.1em, pos=0.3] {f} (s3)
          (s2) edge node [xshift=-0.1em, pos=0.3] {e} (s4)
          (s1) edge node [right, xshift=0.1em, pos=0.3] {s} (s5)
          (s4) edge [line width=1pt] node [right, xshift=0.1em, pos=0.3] {f} (s6)
          (s3) edge node [left, xshift=0.1em, pos=0.3] {e} (s6)
          (s1) edge [out=270, in=150] node [xshift=0.1em, pos=0.5] {e} (s7)
          (s5) edge node [xshift=0.1em, pos=0.45] {e} (s7)
          (s3) edge node [below, xshift=-0.1em, pos=0.45] {s} (s10)
          (s5) edge [out=330, in=30] node [right, xshift=-0.1em, pos=0.5] {r} (s10.east)
          (s7) edge node [pos=0.5] {r} (s8)
          (s3) edge node [xshift=0.1em, pos=0.50] {e} (s8)
          (s6) edge [out=315, in=150] node [below, xshift=-0.25em, pos=0.4] {s} (s8)
          (s6) edge node [pos=0.3] {e} (s9)
          (s10) edge node [below=-0.1em, pos=0.5] {e} (s8)
          (s8) edge node [below=-0.1em, pos=0.5] {e} (s9);
      \end{tikzpicture}
    }
  \caption{Local and global states of a component-based system}\label{fig:lts}
\end{figure}

\ldots



  %!TEX root=main.tex
\chapter{Background}\label{ch:background}

This chapter presents an overview of the background literature and state-of-the-art in the field of \RV for distributed systems. 
%
We first give a brief synopsis of \RV, and define the terms introduced in the preceding chapter.
%
This is followed by an account of distributed systems that acquaints readers with the concepts used throughout this report.
%
Finally, a series of recent works is discussed, comparing and contrasting similarities and differences between them.
%
This exposition should reassert the current limitations in the area, and in doing so, underscore the novelty of our research contributions proposed in \cref{ch:introduction}.

\ldots

  % Add bibliography.
  \bibliographystyle{abbrv}
  \bibliography{bibliography}

  \appendix
  %!TEX root=main.tex
\chapter{Trace Partitioning and Local Monitoring for Asynchronous Components}\label{ch:appendix}

The paper entitled \quot{Trace Partitioning and Local Monitoring for Asynchronous Components} was published in SEFM 2017 under Springer LNCS.

\dots

\end{document}