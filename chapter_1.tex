%!TEX root=main.tex
\chapter{Introduction}\label{ch:introduction}

Numerous software systems are nowadays architected in terms of \emph{asynchronous components}~\cite{Chappell2004,Josuttis2007,AghaMST97} that execute independently to one another without recourse to a global clock or shared state.
%
Instead, components interact together via well-defined interfaces and non-blocking messaging~\cite{HohpeWoolf2003} to create dynamic and loosely-coupled software organisations.
%
Such architectures facilitate incremental updates, tolerate independent component failures and permit the various units of execution to be \emph{distributed} across different locations~\cite{Garg2014,DollimoreKindbergCoulouris2005}. 
%
Despite their advantages, these systems are notoriously hard to design, and even harder to program and get right, and ensuring their correctness in terms of their expected behaviour becomes paramount.

\section{Distributed Runtime Verification}

While distributed systems inherit the characteristics inherent to asynchronous settings, their execution is further complicated due to physical constraints, such as the lack of a global clock and possibility of independent failures~\cite{Ghosh2014,DollimoreKindbergCoulouris2005}.
%
In the local asynchronous case, traditional pre-deployment verification techniques such as model checking and testing~\cite{ClarkeGrumbergPeled1999,MyersSandlerBadgett2011} often \emph{scale poorly} because the set of execution paths considered is invariably dwarfed by the vast number of possible execution paths of the system.
%
In a distributed scenario, use of these verification approaches is often problematic, if not \emph{impractical}, due to the aforementioned complications. 

\emph{Runtime Verification} (\RV)~\cite{LeuckerS09,FalconeFM12} is complementary approach that evades some of the limitations of pre-deployment techniques by deferring the analysis until runtime.
%
It employs \emph{monitors}
%
to \emph{incrementally} analyse the system's behaviour (exhibited as a sequence of \emph{trace} events) up to the current execution point, to determine whether a correctness specification under investigation is satisfied or violated.

\ldots
 
\begin{example}\label{eg:local}
  Consider the system \kSys composed of processes \kP and \kQ, given in terms of the \emph{labelled transition systems} in \cref{fig:lts}.
  %
  \kP can initially fork (denoted by action \actiont{f}) a process, after which it either sends (\actiont{s}) a message and exits (\actiont{e}), or exits (\actiont{e}) immediately; \kQ is a simpler process that performs a single receive (\actiont{r}) and exits (\actiont{e}).
  %
  A possible correctness property states that \quot{\kSys does not fork processes at startup}.

  \medskip
  When \kSys exhibits the witness trace \trace{f,r,e,e}, the monitor can detect a violation of this property.
  %
  For a different execution interleaving, \eg \trace{r,e,f,e} (where \trace{f} is not the first event), the typical \RV analysis would be unable to detect the fact that \kSys is capable of performing \actiont{f}.
  %
  Recouping this \emph{lack of precision} is possible, but this would require the specification to consider \emph{all} the possible trace event permutations that the composition of \kP and \kQ may exhibit (\cref{fig:lts_sys}). 
  %
  One easily observes that adding new components to \kSys aggravates the specification task to the point where it becomes unwieldy and error-prone.
  Reformulating the original property to consider \kP in isolation, \ie \quot{\kP does not fork processes at startup}, eliminates the need to account for the behaviour of other nonrelevant components.
\end{example}

\begin{figure}[b!]
  \pframe{
    \centering
    \subfloat[The labelled transition systems for processes \kP and \kQ\label{fig:lts_pq}]{
      \begin{tikzpicture}[>=stealth', shorten >=0.05em, shorten <=0.05em, auto, node distance=2em]
        \draw[shift={(-5.5em,0)}, color=white] (0em,0em) rectangle (16em,1em);

        % Draw process P.
        \node[point] (p0) {};
        \node[point, below=of p0] (p1) {};
        \node[point, below=of p1, xshift=2em] (p2) {};
        \node[point, below=of p1, xshift=-2em] (p3) {};
        \node[label, above=0 of p0]{\textcolor{white}{\kQ}\!\!\!\!\!\kP};

        % Draw process Q.
        \node[point, right=6em of p0] (q0) {};
        \node[point, below=of q0] (q1) {};
        \node[point, below=of q1] (q2) {};
        \node[label, above=0 of q0]{\kQ};

        % Draw transitions between P nodes.
        \path[-stealth', font=\tiny\ttfamily, left]
          (p0) edge [line width=1pt] node [xshift=0.1em] {f} (p1)
          (p1) edge node [right, xshift=0.1em, pos=0.3] {e} (p2)
          (p1) edge node [pos=0.3] {s} (p3)
          (p3) edge [out=330, in=210] node [below] {e} (p2);
        
        % Draw transitions between Q nodes.
        \path[-stealth', font=\tiny\ttfamily, left]
          (q0) edge node [xshift=0.1em] {r} (q1)
          (q1) edge node [xshift=0.1em] {e} (q2);
      \end{tikzpicture}
    }\qquad\qquad
    \subfloat[System \kSys as a composition of processes \kP and \kQ\label{fig:lts_sys}]{
      \begin{tikzpicture}[>=stealth', shorten >=0.05em, shorten <=0.05em, auto, node distance=2em]
        \draw[shift={(9em,0)}, color=white] (0em,0em) rectangle (16em,1em);

        % Draw composed system.
        \node[point, right=10em of q0] (s0) {};
        \node[point, below=of s0, xshift=2em] (s1) {};
        \node[point, below=of s0, xshift=-2em] (s2) {};
        \node[point, below=of s2, xshift=2em] (s3) {};
        \node[point, below=of s2, xshift=-2em] (s4) {};
        \node[point, below=of s1, xshift=2em] (s5) {};
        \node[point, below=of s3, xshift=-2em] (s6) {};
        \node[point, below=of s5] (s7) {};
        \node[point, below=2em of s6, xshift=2em] (s8) {};
        \node[point, below=4em of s6, xshift=-2em] (s9) {};
        \node[point, below=of s7] (s10) {};
        \node[label, above=0 of s0]{$\kSys=\kP\!\mid\!\kQ$};

        % Draw transitions between system nodes.
        \path[-stealth', font=\tiny\ttfamily, left]
          (s0) edge [line width=1pt] node [right, xshift=0.1em, pos=0.3] {f} (s1)
          (s0) edge node [pos=0.3] {r} (s2)
          (s1) edge node [pos=0.3] {r} (s3)
          (s2) edge [line width=1pt] node [right, xshift=0.1em, pos=0.3] {f} (s3)
          (s2) edge node [xshift=-0.1em, pos=0.3] {e} (s4)
          (s1) edge node [right, xshift=0.1em, pos=0.3] {s} (s5)
          (s4) edge [line width=1pt] node [right, xshift=0.1em, pos=0.3] {f} (s6)
          (s3) edge node [left, xshift=0.1em, pos=0.3] {e} (s6)
          (s1) edge [out=270, in=150] node [xshift=0.1em, pos=0.5] {e} (s7)
          (s5) edge node [xshift=0.1em, pos=0.45] {e} (s7)
          (s3) edge node [below, xshift=-0.1em, pos=0.45] {s} (s10)
          (s5) edge [out=330, in=30] node [right, xshift=-0.1em, pos=0.5] {r} (s10.east)
          (s7) edge node [pos=0.5] {r} (s8)
          (s3) edge node [xshift=0.1em, pos=0.50] {e} (s8)
          (s6) edge [out=315, in=150] node [below, xshift=-0.25em, pos=0.4] {s} (s8)
          (s6) edge node [pos=0.3] {e} (s9)
          (s10) edge node [below=-0.1em, pos=0.5] {e} (s8)
          (s8) edge node [below=-0.1em, pos=0.5] {e} (s9);
      \end{tikzpicture}
    }
  }
  \caption{Local and global states of a component-based system}\label{fig:lts}
\end{figure}

\ldots

