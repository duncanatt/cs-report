%!TEX root=main.tex
\documentclass[11pt,a4paper]{csreport}


%:::::::::::::::::::::::::::::::::::::::::::::::::::::::::::::::::::::::::::::::
% Package imports with basic inlined configuration.
%:::::::::::::::::::::::::::::::::::::::::::::::::::::::::::::::::::::::::::::::

% Provides support for drawing vector-based figures.
\usepackage{tikz}

% Provides support for handling floating figures.
\usepackage{float}

% Provides support for handling sub floating figures.
\usepackage{subfig}

% Provides additional mathematical fonts and symbols.
\usepackage{amsfonts}

% Provides support for mathematical formulae.
\usepackage{amsmath}

% Provides additional mathematical symbols (such as black square, etc.).
\usepackage{amssymb}

% Provides support for defining mathematical theorems and definitions.
\usepackage{amsthm}

% Provides extended support for the array and tabular environments for optional 
% column formats.
\usepackage{array}

% Provides support for multirow tables.
\usepackage{multirow}

% Provides miscellaneous mathematical symbols (such as denotation brackets).
\usepackage{stmaryrd}

% Provides support for modifying standard enumerations (such as inline lists,
% roman numerals in lists, etc.).
\usepackage[inline]{enumitem}

% Provides miscellaneous dingbats (such as check marks, ticks, etc.).
\usepackage{pifont}

% Provides support for low-level string processing (such as string chopping).
\usepackage{xstring}

% Provides support for including and displaying code listings for various 
% languages.
\usepackage{listings}

% Provides support for typesetting algorithm pseudo code. Automatically imports
% Algorithmicx package, since Algpseudocode is a layout for Algorithmicx.
\usepackage[end]{algpseudocode}

% Provides support for displaying algorithms as floats.
\usepackage[plain]{algorithm}

% Provides support for generating hyperlinks for in-document cross-referencing
% and web URLs.
\usepackage[linktocpage=true, hypertexnames=false]{hyperref}

% Provides support for automatically including caption labels with references.
% Cleveref should be always included AFTER Hyperref!
\usepackage[nameinlink]{cleveref}


%:::::::::::::::::::::::::::::::::::::::::::::::::::::::::::::::::::::::::::::::
% Additional package-specific configuration.
%:::::::::::::::::::::::::::::::::::::::::::::::::::::::::::::::::::::::::::::::

% Configure Cleveref label names.
\newcommand{\crefrangeconjunction}{--}
\crefname{figure}{figure}{figures}
\crefname{example}{example}{examples}
\crefname{section}{section}{sections}
\crefname{chapter}{chapter}{chapters}
\crefname{table}{table}{tables}
\crefname{appendix}{appendix}{appendices}

% Configure Hyperref colours.
\hypersetup{
  colorlinks,
  linkcolor=darkred,
  citecolor=darkgreen,
  urlcolor=darkblue
}

% Includes.
%:::::::::::::::::::::::::::::::::::::::::::::::::::::::::::::::::::::::::::::::
% Math lettering shorthand.
%:::::::::::::::::::::::::::::::::::::::::::::::::::::::::::::::::::::::::::::::

% Shorthand for math calligraphic letters.
\newcommand{\cA}{\ensuremath{\mathcal{A}}\xspace}
\newcommand{\cB}{\ensuremath{\mathcal{B}}\xspace}
\newcommand{\cC}{\ensuremath{\mathcal{C}}\xspace}
\newcommand{\cD}{\ensuremath{\mathcal{D}}\xspace}
\newcommand{\cE}{\ensuremath{\mathcal{E}}\xspace}
\newcommand{\cF}{\ensuremath{\mathcal{F}}\xspace}
\newcommand{\cG}{\ensuremath{\mathcal{G}}\xspace}
\newcommand{\cH}{\ensuremath{\mathcal{H}}\xspace}
\newcommand{\cI}{\ensuremath{\mathcal{I}}\xspace}
\newcommand{\cJ}{\ensuremath{\mathcal{J}}\xspace}
\newcommand{\cK}{\ensuremath{\mathcal{K}}\xspace}
\newcommand{\cL}{\ensuremath{\mathcal{L}}\xspace}
\newcommand{\cM}{\ensuremath{\mathcal{M}}\xspace}
\newcommand{\cN}{\ensuremath{\mathcal{N}}\xspace}
\newcommand{\cO}{\ensuremath{\mathcal{O}}\xspace}
\newcommand{\cP}{\ensuremath{\mathcal{P}}\xspace}
\newcommand{\cQ}{\ensuremath{\mathcal{Q}}\xspace}
\newcommand{\cR}{\ensuremath{\mathcal{R}}\xspace}
\newcommand{\cS}{\ensuremath{\mathcal{S}}\xspace}
\newcommand{\cT}{\ensuremath{\mathcal{T}}\xspace}
\newcommand{\cU}{\ensuremath{\mathcal{U}}\xspace}
\newcommand{\cV}{\ensuremath{\mathcal{V}}\xspace}
\newcommand{\cW}{\ensuremath{\mathcal{W}}\xspace}
\newcommand{\cX}{\ensuremath{\mathcal{X}}\xspace}
\newcommand{\cY}{\ensuremath{\mathcal{Y}}\xspace}
\newcommand{\cZ}{\ensuremath{\mathcal{Z}}\xspace}

% Shorthand for math sans-serif letters.
\newcommand{\sA}{\ensuremath{\mathsf{A}}\xspace}
\newcommand{\sB}{\ensuremath{\mathsf{B}}\xspace}
\newcommand{\sC}{\ensuremath{\mathsf{C}}\xspace}
\newcommand{\sD}{\ensuremath{\mathsf{D}}\xspace}
\newcommand{\sE}{\ensuremath{\mathsf{E}}\xspace}
\newcommand{\sF}{\ensuremath{\mathsf{F}}\xspace}
\newcommand{\sG}{\ensuremath{\mathsf{G}}\xspace}
\newcommand{\sH}{\ensuremath{\mathsf{H}}\xspace}
\newcommand{\sI}{\ensuremath{\mathsf{I}}\xspace}
\newcommand{\sJ}{\ensuremath{\mathsf{J}}\xspace}
\newcommand{\sK}{\ensuremath{\mathsf{K}}\xspace}
\newcommand{\sL}{\ensuremath{\mathsf{L}}\xspace}
\newcommand{\sM}{\ensuremath{\mathsf{M}}\xspace}
\newcommand{\sN}{\ensuremath{\mathsf{N}}\xspace}
\newcommand{\sO}{\ensuremath{\mathsf{O}}\xspace}
\newcommand{\sP}{\ensuremath{\mathsf{P}}\xspace}
\newcommand{\sQ}{\ensuremath{\mathsf{Q}}\xspace}
\newcommand{\sR}{\ensuremath{\mathsf{R}}\xspace}
\newcommand{\sS}{\ensuremath{\mathsf{S}}\xspace}
\newcommand{\sT}{\ensuremath{\mathsf{T}}\xspace}
\newcommand{\sU}{\ensuremath{\mathsf{U}}\xspace}
\newcommand{\sV}{\ensuremath{\mathsf{V}}\xspace}
\newcommand{\sW}{\ensuremath{\mathsf{W}}\xspace}
\newcommand{\sX}{\ensuremath{\mathsf{X}}\xspace}
\newcommand{\sY}{\ensuremath{\mathsf{Y}}\xspace}
\newcommand{\sZ}{\ensuremath{\mathsf{Z}}\xspace}

% Shorthand for math blackboard bold letters (requires amsfonts package).
\newcommand{\bA}{\ensuremath{\mathbb{A}}\xspace}
\newcommand{\bB}{\ensuremath{\mathbb{B}}\xspace}
\newcommand{\bC}{\ensuremath{\mathbb{C}}\xspace}
\newcommand{\bD}{\ensuremath{\mathbb{D}}\xspace}
\newcommand{\bE}{\ensuremath{\mathbb{E}}\xspace}
\newcommand{\bF}{\ensuremath{\mathbb{F}}\xspace}
\newcommand{\bG}{\ensuremath{\mathbb{G}}\xspace}
\newcommand{\bH}{\ensuremath{\mathbb{H}}\xspace}
\newcommand{\bI}{\ensuremath{\mathbb{I}}\xspace}
\newcommand{\bJ}{\ensuremath{\mathbb{J}}\xspace}
\newcommand{\bK}{\ensuremath{\mathbb{K}}\xspace}
\newcommand{\bL}{\ensuremath{\mathbb{L}}\xspace}
\newcommand{\bM}{\ensuremath{\mathbb{M}}\xspace}
\newcommand{\bN}{\ensuremath{\mathbb{N}}\xspace}
\newcommand{\bO}{\ensuremath{\mathbb{O}}\xspace}
\newcommand{\bP}{\ensuremath{\mathbb{P}}\xspace}
\newcommand{\bQ}{\ensuremath{\mathbb{Q}}\xspace}
\newcommand{\bR}{\ensuremath{\mathbb{R}}\xspace}
\newcommand{\bS}{\ensuremath{\mathbb{S}}\xspace}
\newcommand{\bT}{\ensuremath{\mathbb{T}}\xspace}
\newcommand{\bU}{\ensuremath{\mathbb{U}}\xspace}
\newcommand{\bV}{\ensuremath{\mathbb{V}}\xspace}
\newcommand{\bW}{\ensuremath{\mathbb{W}}\xspace}
\newcommand{\bX}{\ensuremath{\mathbb{X}}\xspace}
\newcommand{\bY}{\ensuremath{\mathbb{Y}}\xspace}
\newcommand{\bZ}{\ensuremath{\mathbb{Z}}\xspace}


%:::::::::::::::::::::::::::::::::::::::::::::::::::::::::::::::::::::::::::::::
% Math object definitions.
%:::::::::::::::::::::::::::::::::::::::::::::::::::::::::::::::::::::::::::::::

% Operators.
\DeclareMathOperator{\righttriplearrows} {{\; \tikz{ \foreach \y in {0, 0.1, 0.2} { \draw [-stealth] (0, \y) -- +(0.5, 0);}} \; }}
\newcommand{\Righttriplearrows} {{\; \tikz{ \foreach \y in {0, 0.1, 0.2} { \draw [-stealth] (0, \y) -- +(0.5, 0);}} \; }}

% Actions and traces.
\makeatletter
\newcommand{\actiont}[1]{%
	\@tempswafalse
	\@for\next:=#1\do
	{\if@tempswa.\else\@tempswatrue\fi\texttt{\next}}%
}
\newcommand{\trace}[1]{\textit{\actiont{#1}}}
\makeatother

% Process constants.
\newcommand{\kP}[1][]{\if\relax\detokenize{#1}\relax\ensuremath{P}\else\ensuremath{P_{#1}}\fi\xspace}
\newcommand{\kQ}[1][]{\if\relax\detokenize{#1}\relax\ensuremath{Q}\else\ensuremath{Q_{#1}}\fi\xspace}
\newcommand*{\kSys}{\ensuremath{\textsc{Sys}}\xspace}


%:::::::::::::::::::::::::::::::::::::::::::::::::::::::::::::::::::::::::::::::
% Environments.
%:::::::::::::::::::::::::::::::::::::::::::::::::::::::::::::::::::::::::::::::

% Define the general environment 'QED' marker.
\newcommand*{\envqed}{\hfill$\blacksquare$}

% Checks if a counter exists.
\makeatletter
\newcommand*\ifcounter[1]{%
  \ifcsname c@#1\endcsname
    \expandafter\@firstoftwo
  \else
    \expandafter\@secondoftwo
  \fi
}

\ifcounter{chapter}{
  
  % Example environment.
  \newcounter{example}[chapter]
  \renewcommand{\theexample}{\thechapter.\arabic{example}}
  \newenvironment{example}[1][]{
    \refstepcounter{example}\bigskip
    \if\relax\detokenize{#1}\relax
      \noindent\textbf{Example~\theexample}.
    \else
      \noindent\textbf{Example~\theexample} (#1).
    \fi}
  {
    \envqed\bigskip
  }

  % Definition environment.
  \newcounter{definition}[chapter]
  \renewcommand{\thedefinition}{\thechapter.\arabic{definition}}
  \newenvironment{definition}[1][]{
    \refstepcounter{definition}\bigskip
    \if\relax\detokenize{#1}\relax
      \noindent\textbf{Definition~\thedefinition}.
    \else
      \noindent\textbf{Definition~\thedefinition} (#1).
    \fi}
  {
    \envqed\bigskip
  }

  % Theorem environment.
  \newcounter{theorem}[chapter]
  \renewcommand{\thetheorem}{\thechapter.\arabic{theorem}}
  \newenvironment{theorem}[1][]{
    \refstepcounter{theorem}\bigskip
    \if\relax\detokenize{#1}\relax
      \noindent\textbf{Theorem~\thetheorem}.
    \else
      \noindent\textbf{Theorem~\thetheorem} (#1).
    \fi}
  {
    \envqed\bigskip
  }
}
\makeatother
%!TEX root=../main.tex

% Abbreviations.
\newcommand*{\eg}{\textit{e.g.}\xspace}
\newcommand*{\Eg}{\textit{E.g.}\xspace}
\newcommand*{\ie}{\textit{i.e.,}\xspace}
\newcommand*{\etc}{\textit{etc.}\xspace}
\newcommand*{\corr}{\textit{corr.}\xspace}
\newcommand*{\etal}{\textit{et~al.}\xspace}
\newcommand*{\wrt}{w.r.t.\xspace}
\newcommand*{\cf}{\textit{cf.}\xspace}
\newcommand*{\nb}{\textit{n.b.}\xspace}
\newcommand*{\resp}{resp.\xspace}
\newcommand*{\vs}{\textit{vs.}\xspace}

% Quotations.
\newcommand{\quot}[1]{\textit{``#1''}}

% Acronyms.
\newcommand*{\PID}{PID\xspace}
\newcommand*{\RV}{RV\xspace}
\newcommand*{\LTL}{\textsc{LTL}\xspace}
\newcommand*{\PtDTL}{\textsc{PtDTL}\xspace}
\newcommand*{\DTL}{\textsc{DTL}\xspace}
\newcommand*{\LTLIII}{\ensuremath{\textsc{LTL}_3}\xspace}
\newcommand*{\LTLK}{\ensuremath{\textsc{LTL}_{2k+4}}\xspace}
\newcommand*{\CTL}{CTL\xspace}
\newcommand*{\muHML}{\ensuremath{\mu}HML\xspace}
\newcommand*{\mHML}{\textsc{mHML}\xspace}
\newcommand*{\sHML}{\textsc{sHML}\xspace}
\newcommand*{\DATE}{\textsc{DATE}\xspace}
\newcommand*{\MtTL}{\textsc{MtTL}\xspace}
\newcommand*{\MTL}{\textsc{MTL}\xspace}
\newcommand*{\muCalculus}{\ensuremath{\mu}-calculus\xspace}
\newcommand*{\AOP}{AOP\xspace}
\newcommand*{\Erlang}{Erlang\xspace}
\newcommand*{\BIF}{BIF\xspace}
%!TEX root=../main.tex

%:::::::::::::::::::::::::::::::::::::::::::::::::::::::::::::::::::::::::::::::
% Additional TikZ configuration.
%:::::::::::::::::::::::::::::::::::::::::::::::::::::::::::::::::::::::::::::::

% Imported TikZ libraries.
\usetikzlibrary {
  matrix,
  shapes,
  arrows,
  shadows,
  calc,
  chains,
  decorations.pathmorphing,
  decorations.text,
  arrows.meta,
  patterns,
  fit
}


%:::::::::::::::::::::::::::::::::::::::::::::::::::::::::::::::::::::::::::::::
% Graphic objects.
%:::::::::::::::::::::::::::::::::::::::::::::::::::::::::::::::::::::::::::::::

% Custom shapes.
\tikzset{
  label/.style={
    font=\scriptsize\itshape,
    inner sep=0.4em
  },
  point/.style={
    circle,
    fill=black,
    text width=0.3em,
    inner sep=0
  },
  state/.style={
    circle,
    text width=0.8em,
    inner sep=0.1em,
    text depth=0.08em,
    draw,
    font=\scriptsize
  }
}